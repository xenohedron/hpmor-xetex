\documentclass[10pt,extrafontsizes,twoside,openany,final]{memoir}

% THIS DOCUMENT REQUIRES XeLaTeX TO COMPILE!

\usepackage{lettrine}		% Used for the fancy caps at the start of each chapter
\usepackage{xspace}		    % Takes care of spaces after macros
\usepackage{amsmath}		% Provides the align environment, used in chapter 13 for the notes
\usepackage[protrusion=basictext]{microtype}
                            % Makes right margin neater by letting certain punctuation protrude slightly
\usepackage{fontspec}		% For the many fonts
\usepackage{xunicode}       % Probably needed for something
\usepackage{xstring}        % Needed for something
\usepackage{censor}         % Used in "Time Pressure" to redact Hermione's name
\usepackage[useregional]{datetime2}
                            % Used for the build timestamp in the colophon

\usepackage[bookmarks=true,unicode=true,pdfborder={0 0 0},
	pdftitle={Harry Potter and the Methods of Rationality},
	pdfauthor={LessWrong}, breaklinks={true},
	pdfkeywords={Harry Potter, rationality},pdfencoding=auto
]{hyperref}                 % Hyperlinks and bookmarks in PDF

\usepackage[autostyle=false, style=english]{csquotes}
\MakeOuterQuote{"}	        % Automatic curly quotes

\usepackage{fmtcount}       % Provides \NUMBERstringnum
\usepackage{graphicx}       % for \reflectbox in header

\input{ushyphex}            % US Hyphenation exceptions

%
% Set-up page sizes and margins
%
\setstocksize{9in}{6in}
\settrimmedsize{\stockheight}{\stockwidth}{*}
\settypeblocksize{7.34in}{4.31in}{*}
\setlrmargins{*}{*}{0.65}
\setulmargins{*}{*}{0.8}
\setheadfoot{\topskip + \baselineskip}{2\baselineskip}
\checkandfixthelayout[fixed]
\fixdvipslayout % fix for xelatex

%
% Font setup: URW Garamond No. 8
%
\setmainfont[
, Ligatures={Common,TeX}
, UprightFont=*-Regular
, ItalicFont=*-Italic
, BoldFont=*-Medium
, BoldItalicFont=*-Medium-Italic
]{GaramondNo8}
%
% Chapters, headings, etc.: Lumos (extra spacing looks much better)
\newfontface\hp[ExternalLocation, LetterSpace=18.0, WordSpace=1.5]{Lumos}
\newcommand{\lumos}[1]{{\hp\MakeUppercase{#1}}}
% Chapter numbers: Lumos (extreme spacing for effect)
\newfontface\hpchap[ExternalLocation, LetterSpace=80.0, WordSpace=2.0]{Lumos}
%
% Font used to draw the magic star
\newfontface\starfont[ExternalLocation]{Miscelanea.ttf}
%
% Draws a “magic star”, used as a decoration
\newcommand*{\Star}{{\starfont{}*}}
%
% Draws three “magic stars”, used as a decoration everywhere
\newcommand*{\Stars}{{\Large\Star\kern-.6ex\lower1.1ex\hbox{\Large\Star}%
    \kern-.1ex\raise.1ex\hbox{\small\Star}\spacefactor1000}}

%
% Section break with three spaced stars
%
\newcommand{\sbreak}{\par\vskip 0pt plus .5\baselineskip%
    {\begin{center} \Star\qquad\Star\qquad\Star \end{center}}\vskip 0pt plus .5\baselineskip \par\noindent}

%
% Simple linebreak
%
\newcommand{\lbreak}{\vspace*{\baselineskip}}

%
% Custom chapter style
%
\makechapterstyle{evans}{%
	\renewcommand*{\chapnamefont}{\hpchap\normalsize}
	\renewcommand*{\chapnumfont}{\chapnamefont\normalsize}
	\renewcommand*{\chaptitlefont}{\hp\large}
	
	\setlength{\beforechapskip}{0pt}
	\setlength{\midchapskip}{0pt}
	\setlength{\afterchapskip}{1\baselineskip}
	
	\renewcommand*{\printchapternum}{% " C H A P T E R   O N E "
		\begin{center} \chapnumfont \hyperref[contents]{CHAPTER \NUMBERstringnum{\thechapter}}\end{center}}

	\renewcommand*{\printchaptername}{}% Not needed, "CHAPTER" added above
	
	\renewcommand*{\printchaptertitle}[1]{% "A DAY OF VERY LOW PROBABILITY"
		\vskip 1cm \begin{center}\chaptitlefont \MakeUppercase{##1}\end{center}\par \vskip 1cm}
	
	\renewcommand*{\chaptermark}[1]{% "CHAPTER ONE" and "A DAY OF..."
		\markboth{\MakeUppercase{##1}}{%
		  \MakeUppercase{\chaptername}~\protect\NUMBERstringnum{\thechapter}}}

	\renewcommand*{\tocmark}{\markboth{}{\MakeUppercase{Contents}}}
	
	\renewcommand{\tocheadstart}{\chapterheadstart}
	\renewcommand{\aftertoctitle}{\thispagestyle{empty}\afterchaptertitle}
}
\chapterstyle{evans}% Actually implements the above chapter style code

\newcommand*{\hpmorchapbreak}{\linebreak} % Allows the break to be disabled in TOC
\newcommand{\wrapchapter}[2]{\chapter[% Custom linebreak for long chapter titles
        \texorpdfstring{%
        #1\protect\hpmorchapbreak #2}{% Header
        #1 #2}]{% Bookmark for PDF
        #1\protect\hpmorchapbreak #2}}% First page of chapter


%
% Subsection
%
\setsubsecheadstyle{\bfseries\scshape} % Subsection titles in small caps
\beforesubsecskip=1.5\baselineskip % skip before the heading
\aftersubsecskip=.5\baselineskip plus .5\baselineskip % skip after the heading
\setsubsechook{\nopagebreak\vskip 0pt plus 3\baselineskip}

%
% Configure line spacing and paragraph spacing, and modifying penalties
%
\linespread{1.17} % Wider line spacing looks much better
\raggedbottom     % Better to have consistent line spacing than vertically justified text
% \sloppybottom     % Allows widow lines to be added to the bottom of the page.
\frenchspacing    % Single space after a period
%
\setlength{\emergencystretch}{.07\textwidth}
% If a paragraph cannot be typeset without compromises, try wider line spacing
%
\clubpenalty=80 % Mildly discourages first line of paragraph on previous page
\widowpenalty=9000 % Avoids last line of paragraph on next page
\brokenpenalty=2000 % Avoids hyphenation across a page break
\hyphenpenalty=60 % Penalty for automatic hyphenation
\exhyphenpenalty=75 % Penalty for breaking after a dash or explicit hyphen
\adjdemerits=80000 % Penalty for visually incompatible lines
\doublehyphendemerits=70000 % Avoids hyphenation on consecutive lines
\hbadness=4000 % When compiling, only reports underfull lines if badness>4000

%
% Adjust space around lists (not entirely sure why)
%
\setlength{\topsep}{.5\baselineskip plus 1\baselineskip minus .5\baselineskip}
\setlength{\partopsep}{.5\baselineskip plus 1\baselineskip minus .5\baselineskip}

%
% Lettrine font pick and configuration
%
\renewcommand{\LettrineFontHook}{\hp} % Uses Lumos for the large letter
\renewcommand{\LettrineTextFont}{} % Normal font for the rest of the first word
% For when the first word is in italics, use this command instead
\newcommand{\lettrinemph}[3][]{\lettrine[#1]{#2}{\emph{#3}}}
\setcounter{DefaultLines}{1}
% Large cap instead of drop cap because of many short first lines
\renewcommand{\DefaultLoversize}{0}
\renewcommand{\DefaultLraise}{0}

%
% Page numbering, footer, header (confusing)
%
\newcommand{\pageInFooter}{{\small\Star\ \makebox[2em][c]{\thepage\,}\Star}}
% Page number centered between two small stars in footer
% Same on all pages in body text
\makeevenfoot{plain}{}{\pageInFooter}{}
\makeoddfoot{plain}{}{\pageInFooter}{}
\makeevenfoot{headings}{}{\pageInFooter}{}
\makeoddfoot{headings}{}{\pageInFooter}{}
%
% Left side header: chapter number
% Right side header: chapter name
% Decorative stars in corners
\makeevenhead{headings}{\Stars}{
	\hp\hyperref[contents]{\small\rightmark}}{\reflectbox{\Stars}}
\makeatletter
\makeoddhead{headings}{\Stars}{\parbox{99mm}{\centering\hp\small\leftmark}}{\reflectbox{\Stars}}
\copypagestyle{cleared}{empty}
\makepsmarks{headings}{%
\createmark{chapter}{right}{shownumber}{\@chapapp\ }{. \ }}
\makeatother

%
% XeTeX character classes (hacks)
%
\newXeTeXintercharclass \punctuationClass
\XeTeXcharclass `\’ \punctuationClass
\XeTeXcharclass `\‘ \punctuationClass
\XeTeXcharclass `\“ \punctuationClass
\XeTeXcharclass `\” \punctuationClass
\XeTeXcharclass `\. \punctuationClass
\XeTeXcharclass `\, \punctuationClass
\XeTeXcharclass `\: \punctuationClass
\XeTeXcharclass `\? \punctuationClass
\XeTeXcharclass `\! \punctuationClass
\XeTeXcharclass `\: \punctuationClass
%
\newXeTeXintercharclass \digitClass
\XeTeXcharclass `\0 \digitClass
\XeTeXcharclass `\1 \digitClass
\XeTeXcharclass `\2 \digitClass
\XeTeXcharclass `\3 \digitClass
\XeTeXcharclass `\4 \digitClass
\XeTeXcharclass `\5 \digitClass
\XeTeXcharclass `\6 \digitClass
\XeTeXcharclass `\7 \digitClass
\XeTeXcharclass `\8 \digitClass
\XeTeXcharclass `\9 \digitClass
%
\newXeTeXintercharclass \dashClass
\XeTeXcharclass `\— \dashClass % em
\XeTeXcharclass `\– \dashClass % en
%
\newXeTeXintercharclass \hyphenClass
\XeTeXcharclass `\- \hyphenClass % hyphen
%
\XeTeXinterchartokenstate = 1 % enable functionality
%
% Automatic spacing around em dashes, allows line breaks after
\XeTeXinterchartoks \dashClass 0 = {\discretionary{}{}{\,}}
\XeTeXinterchartoks 0 \dashClass = {\,}
\XeTeXinterchartoks \dashClass 255 = {\discretionary{}{}{\,}}
\XeTeXinterchartoks 255 \dashClass = {\,}

%
% Style definitions and hacks
%
% In the text, em dashes are encoded explicitly
% and the ellipsis is encoded with {\el}
\newcommand*{\el}{\,.\,.\,.} % Ellipsis definition
%
% Logical formatting macros
\newcommand*{\parsel}[1]{\textit{#1}} % Parseltongue
\newcommand*{\prophecy}[1]{\textbf{\textit{\textsc{#1}}}} % Prophecies
\newcommand*{\headline}[1]{\begin{center}\textsc{#1}\end{center}} % Newspaper headlines
\newcommand*{\inlineheadline}[1]{\textsc{#1}} % Newspaper headlines inline with text
%
% Macros to format certain words and abbreviations automatically
\newcommand*{\AM}{{\scshape am}\xspace} % AM in small caps, for time of day
\newcommand*{\PM}{{\scshape pm}\xspace} % PM in small caps, for time of day
%
% Makes a written note in italics (first used, Hogwarts acceptance letter)
\newenvironment{writtenNote}{%\parindent=1cm%
	\vskip 1\baselineskip plus 1\baselineskip minus .5\baselineskip%
	\begin{adjustwidth}{\parindent}{\parindent}\begin{em}%
	\par\noindent}
	{\end{em}\end{adjustwidth}\vskip 1\baselineskip plus 1\baselineskip minus .5\baselineskip}
% Makes a letter address and closing, for use with above writtenNote environment
\newcommand{\letterAddress}[1]{\pagebreak[1]\noindent{}#1\nopagebreak[4]\par}
\newcommand{\letterClosing}[2][\vskip 1\baselineskip]{\nopagebreak[4]#1\par\nopagebreak[5]\noindent#2}
%
% Makes an inscription in small caps (first used, Gringotts)
\newenvironment{inscription}{%\parindent=1cm%
	\vskip 1\baselineskip plus .5\baselineskip minus .5\baselineskip%
	\begin{adjustwidth}{4\parindent}{\parindent}\scshape}
	{\end{adjustwidth}\vskip 1\baselineskip plus .5\baselineskip minus .5\baselineskip}
%
% Not sure what these are included for:
\fboxrule=1pt
\fboxsep=-1pt
%
%
%
% Hyphenation corrections
%
\hyphenation{Gryf-fin-dor Huf-fle-puff Sly-the-rin Wi-zen-ga-mot
McGon-a-gall Dum-ble-dore Malfoy Potter Her-mione Flit-wick
Vol-de-mort Le-strange Gil-deroy Scour-gify Sev-erus
Alo-ho-mora Mun-dun-gus Hogs-meade Nym-pha-dora
Ver-i-ta-serum Tre-lawney Az-ka-ban Poly-juice Sirius basi-lisk
Phi-neas McKin-non Filius Quiri-nus Love-good whom-ever
Gryf-fin-dors Huf-fle-puffs Sly-the-rins}
